We assume that $p$ is less than $n$ i.e. $p < n$, but still relatively large. This assumption is important here because:
\begin{enumerate}
    \item It allows us to use the OLS benchmark for comparisions. That is, OLS being a traditional estimator restricts us for $p$ to be less than $n$ in all cases. Using this restriction of $p < n$ allows us to use the OLS benchmark to undertand the efficiency and robustness of our sparsity estimators. 
    \item By ensuring that $p$ is stil high but less than $n$, we are able to create a high-dimensional scenario where traditional OLS may struggle, yet remain computationally possible. T
\end{enumerate}

This setup allows us to directly compare the performance of OLS against Sparsity-Based Estimators (SBEs) in a setting where $p$ approaches $n$, testing the limits of OLS while also exploring the potential advantages of SBEs. 

\textcolor{red}{Asymptotically normal assump???}
