We study rotation to understand how likely it is to maintain sparsity when the control variables are re-expressed in different ways. \\

Consider a rotation matrix R (a square matrix) which when multiplied by a vector, rotates the vector in space without altering the magnitude of angles between the vectors. The rotation matrix comes with two key properties:
\begin{enumerate}
    \item Orthogonality: $R^{\prime} R=I_p$ where $I_p$ is the identity matrix. This property ensures that the the rotation preserves the lengths of vectors and angles between them, thereby not distorting the data. 
    \item Determinant of 1: $det(R) = 1$. This property implies that the transformation is a pure rotation or reflection without any scaling.
\end{enumerate}


\textbf{Effect of Rotation on Sparsity:}\\
Consider a regression model where the coefficients $\gamma$ are sparse, i.e. most of the coefficients are zero. When the control variables are transformed using a rotation matrix $R$, the new predictors $W_i$ are given by $RW_i$. Correspondingly, the coefficients are also transformed as $\tilde{\gamma}=R \gamma$. This transformed coefficient vector $\tilde{\gamma}$ tends to loose its sparsity because the values get mixed up and spread out, making more coefficients non-zero. 

This effect occurs because rotation redistributes the values across the entire vector space, which dilutes the sparsity present in the original coefficient vector. As a result, the post rotation coefficient vector $\tilde{\gamma}$ is less likely to remain sparse. \\
\\
\textbf{Theorem 1}
\textit{Suppose that the eigenvalues of  $E[W_i W_i\prime]$  are bounded away from zero and infinity, and that  $||\gamma||_2 \approx 1$ . Then the logarithm of the probability that model with regressor $W_i = RW_i$   satisfies the sparsity conditions (sufficiently sparse) is of the order $\frac{-p}{n} \log (p)$.}
\\
\\
\textbf{Interpretation:} This theorem gives a theoretical analysis of the impact of rotations on sparsity. It says that as p increases, the probability that the randomly rotated model will remain sparse also decreases rapidly. Specifically, this probability is given by a bound of the order $p^{-4}log(p)$, which becomes exceedingly small as p grows larger. For instance, when $p>= 50$, the probability of maintaining sparsity is less than $10^{-21}$. Thereby, achieving sparsity through random rotation is highly impractical, particularly in high-dimensional settings where the number of predictors is large. \\
\\
This theorem is based on two conditions:
\begin{enumerate}
    \item The eigenvalues of the matrix $E[W_i W_i\prime]$ must be bounded away from zero and infinity. This condition ensures that variability is appropriate ensuring that there is numerical stability in the model.
    \item The norm of the coefficient vector $\gamma$ should be approximately 1 to ensure the coefficients are of reasonable scale. 
\end{enumerate}

Moreover, this theoretical theoretical analysis presents that the robustness of sparsity-based estimators are heavily influenced by how control variables are transformed. 