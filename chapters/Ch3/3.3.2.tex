THe residual test evaluates the validity of the sparsity assumption by comparing the residual sum of squares (RSS) from the sparsity-based estimators (SBEs) to the Ordinary Least Squares (OLS) regressions. Here, the test uses F-statistic to verify if the difference is significant. F-statistic is used here because it is a standard test to compare variances.  \\
\\
\textbf{Steps Involved:}
\begin{enumerate}
    \item \textbf{Linear Regression Setup:} Consider a linear regression model $Y_i = X_i{\prime}\alpha + \epsilon_i$ where $Y_i$ is the dependent variable, $X_i$ represents the independent variables and $\epsilon_i$ is the error term. Additionally, consider the expectation $E[\epsilon_i | x_i] = 0$ and $\epsilon_i$ is independent across observations conditional on the regressors. 
    \item \textbf{Define the Hypothesis:} \\
    $H_0:$ The difference in RSS between OLS and SBE is not statistically significant\\
    $H_1:$ The difference in RSS between OLS and SBE is statistically significant
    \item \textbf{Assumption of Sparsity:} Introduce a subset $S^* \subset \{1, \dots, p\}$ such that the sparse approximation error $||(I - P_{S}^*)X_\alpha||^2$ is small. Here, $P_S$ is a projection matrix associated with $X_{S^*}$ and $X_\alpha$ represents the fitted values when using the full set of predictors and their associated coefficients.\footnote{The sparse approximation error $||(I - P_{S}^*)X_\alpha||^2$ measures how well the model can be approximated by a sparse set of predictors. Here, $S$ represents the subset of the variables which provides the best sparse approximation. If the projection of $X_\alpha$ onto the subspace spannded by $X_S$ captures most of the variability in the data, the approximation error is small. Essentially, it checks how much of the information is lost when only a sparse set $S$ is used instead of a full set $X$. A small error indicates that the sparse approximation is a good representation of the full model.}
    \item \textbf{Compare RSS:} Calculate and compare the RSS from the SBE model with that of the OLS model. The Residual Sum of Squares (RSS) is calculates as the sum of the squared differences between the observed values $Y_i$ and the predicted values $\hat{Y_i}$ produced by the model. 
    \item \textbf{Use F-statistic to check for the significance:} Compute the F-statistic to test whether the difference in RSS between the models is statisitcally significant.
    If the F-statistic is large, the null hypothesis is rejecting which thereby indicates that the sparisty assumption is not valid since the differences in RSS between the two estimators are significantly different. 
    
    Additionally, the authors present a Lemma (Lemma 3) which provides the theoretical foundations for a residual test, ensuring that under the right conditions, this test can be applied reliably in statistical inference.\footnote{Lemma 3: This lemma validates the assumption of sparsity in a high dimension setting. By showing the test statistic follows a standard normal distribution $N(0,1)$, it allows researchers to apply standard hypothesis procedures. Specifically, if the test statistic aligns with $N(0,1)$, p values can be calculated for statistical inference. This convergence does not confirm that sparsity holds, rather it provides a statistical basis for testing the assumption.Thereby, by comparing the residuals from SBEs and OLS, and observing whether they align with the normal distribution, researchers can determine if the sparsity assumption is reasonable for a particular scenario.}
\end{enumerate}

\textbf{Interpretation:} If the $H_0$ is rejected, the Residual Test finds that the difference between the residual sum of squares of the OLS and SBE estimates are statistically significant. That is, a lower difference suggests that the sparsity assumption holds whereas, a higher difference indigents significant differences in the model indicating the sparsity assumptions may not hold. 

\textcolor{red}{maybe you wanna add something about the conclusion / empirical resutls}
