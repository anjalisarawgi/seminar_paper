The empirical analysis presented in this paper demonstrates the sensitivity of Sparsity-based estimators (SBEs) when utilized with high-dimensional econometric models. The authors test their findings in three empirical studies and demonstrate that even small changes in the control matrix can result in significant fluctuations in SBE estimates. In particular, the findings suggest that even small modifications by using different choices of normalizations may shift the results by more than two standard errors in the estimated values. \\
\\
Additionally, the authors also applied the Hausman and Residual test to these three empirical cases. They frequently observed the rejection of sparsity assumption across the applications in high dimensional scenarios (i.e. when $p \approx n$). This empirical evidence also suggests that while SBEs may offer some efficiency gains when the number of controls approaches the sample size, these gains are often minimal and come at the cost of increased sensitivity and potential bias.\\
\\
This paper shows that SBEs should be used with caution in research. These estimators are popular because they can handle large numbers of controls, but they are not robust in practice because they rely on the sparsity assumption. The authors also advise researchers to practice to justify why sparsity is plausible in their specific applications and consider the potential trade-offs between efficiency and robustness. This paper adds to the discussion on the use of modern econometric techniques, showing that it is important to think carefully about the assumptions behind such methods.