In the recent years, the rapid surge in the availability of large datasets,  combined with advances in statistics, machine learning, and econometrics, has generated significant interest in predictive models with many possible predictors (\cite{illusitionOfSparsity}). This demand has led to the widespread adoption of Sparsity-based Estimators (SBEs), which are particularly well-suited for high-dimensional. Sparsity-based estimators such as the lasso (Least Absolute Shrinkage and Selection Operator) estimator has revolutionalized statistical methods in high-dimensional data analysis, where trational estimators like Ordinary Least Squares (OLS) struggle. \\
\\
The introduction of Lasso by \cite{lasso} performs variable selection by shrinking coefficients towards zero, effectively performing variable selection and regularization simultaneously. Following which, many alternative methods of variable selections were additionally proposed. Few of these include Elastic Net which combines the penalties of Lasso and Ridge Regression (\cite{elasticNet}); SCAD (Smoothly Clipped Absolute Deviation) which aims to reduce the bias introduced by Lasso's penalty(\cite{SCAD}); and various boosting approaches  that incrementally build models to enhance predictive performance (\cite{boostingApproaches}).\\
\\
Despite this success, sparsity based estimators have not been without criticisms. The key assumption underlying these methods is the assumption of sparsity which implies that the true model is sparse. However, recent studies have highlighted the fragility of this assumption. For example, \cite{giannone2021economic} and \cite{wuthrich2023omitted} pointed out that SBEs can be highly sensitive to the choice of the regressor matrix, leading to possibly misleading results. The issue of non-invariance to linear reparametrization, as discussed in the paper "The Fragility of Sparsity", further complicates the use of SBEs. It implies that the results can be highly dependent on the specific setup of the model, thereby undermining robustness of these estimators. \\
\\
The paper "The Fragility of Sparsity" examines the robustness of SBEs through three empirical applications: the effect of abortion on crime (\cite{abortionCrime}), occupational upgrading by Black Southerners (\cite{blackWW2}), and the impact of moral values on voting behaviour (\cite{votingMoral}). The findings reveal that the SBEs are highly sensitive to normalization choices and often fail the sparsity assumption, which raises concerns about the reliability of these estimators in empirical research. The development of tests to assess the validity of the sparsity assumption, as proposed in the paper, represents an important contribution in this field. These tests provides researchers with important tools to verify the robustness of their results when using SBEs under the assumption of sparsity.
