% \textbf{Mean Squared Error Analysis for Machine Learning:}
% These experiments are aimed to evaluate the impact of SBEs within a machine learning context, where the primary focus is the prediction performance of models. For classification tasks, prediction performance can be assessed using metrics like accuracy, while for regression tasks, metrics such as Mean Squared Error (MSE) or R-Squared are commonly used. Our experiments focused on these predictive measures.

% In particular, for the \textbf{Communities and Crime Dataset}, which has relatively high number of observations, we split the dataset into training and testing sets to evaluate the MSE and assess the prediction performance of SBEs compared to OLS.\\
% \\
% \textit{Note:} While communities and crime dataset consists of approximately 1,900 observations, which is smaller than typical datasets used in machine learning evaluations, it was chosen deliberately to test the sparsity assumption in high-dimensional cases. For testing the findings of this paper, it often requires us for our number of predictors to be extremely high, approaching the number of observations $n$ itself. Using a much larger dataset, such as one with 20,000 observations would necessitate increasing p to be around 10,000 or more. This could result in highly redundant datasets due to the excessive number of artificially added features. The communities and Crime dataset thus serves as an ideal compromise, providing sufficient observations to evaluate the prediction performance without introducing undue redundancy.
