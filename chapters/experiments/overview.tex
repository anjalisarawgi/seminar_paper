I utilized these two datasets to provide a comprehensive evaluation of the sparsity
assumption under different empirical conditions. The Lalonde dataset is a well-
known benchmark in causal inference offering a stable environment to validate the
results of our experiments. In contrast, the Communities and Crime Dataset
presents a high-dimensional data scenario with larger number of observations. This
allows us to examine the performance of SBEs under various conditions, including
different subsets of the data for further exploration and splitting the data to compute the mean squared error (MSE) under multiple settings.\\
\\
\textbf{(A) Lalonde Dataset:} The Lalonde dataset looks at how a job training program affects people's earnings. The dataset has information on 445 people. It has 12 variables, including data from the treatment and control groups. This allows us for testing of causal inference methods. Our variables are:
\begin{enumerate}
    \item \textbf{Treatment variable:} The "treat" variable in this dataset treatment variable which shows if someone took part in the training program (1 = yes, 0 = no). It allows us to assess if there is an impact of the job training on the outcome variable. 
    \item \textbf{Outcome variable:} "re78" is the outcome variable for this dataset which represents real earnings in 1989. It measures post treatment earnings to evaluate the impact of the job training program. 
    \item \textbf{Control variables:} The control variables for our model from this dataset include: age, educ, black, hisp, married, nodegr, re74, re75, u74, u75. These variables account for demographic factors (age, educ, black, hisp, married, nodegr), pre-treatment earnings (re74, re75) and employment status before the intervention (u74, u75), allowing us to control for the differences that could impact the outcome. 
\end{enumerate}

\textbf{(B) Communities and Crime Dataset:} The Communities and Crime dataset provides data on various socio-economic, law enforcement and crime related metrics for different communities in the United States. It consists of 2215 observations and 125 features, which allowed for understanding sparsity how a job training program affects people's earnings. Here, each row represents individual communities and municipalities in the United States. The dataset also consists of both categorical and numerical variables, which applies well for evaluating the results of the orignal paper. 
\begin{enumerate}
    \item \textbf{Treatment variable:} The "pop" variable in this dataset treatment variable represents the population of each community. It allows us to assess if there is an impact of the population levels on the crime. 
    \item \textbf{Outcome variable:} "violentPerPop" is the outcome variable which represents the number of crimes per 1,000 people in each community.
    \item \textbf{Control variables:} The control variables in this dataset consist of a wide range of community level indicators which include demographic variables (pop, pctBlack, pctWhite, pctAsian), socio-economic factors (medIncome, pctUnemploy, pctPoverty), variables representing police presence and composition, and other community characteristics.
\end{enumerate}

\textbf{Preprocessing:} The Communities and Crime Dataset had to be preprocessed before we performed experiments to ensure the validity of our results. We begin with handling for missing values. We dropped all columns with missing values more than 50\%. Additionally, we removed all rows with these state codes ['MN', 'MI', 'IL', 'AL', 'NY', 'IA'] due to high number of missing values in these states. We filled the rest of the missing values with the mean values of the respective columns. \\
\\
Next, we checked for outliers and solved for multicollinearity to further refine our data for further analysis. These pre-processing steps significantly reduced the dataset size; the initial shape of the dataset was 2215 rows and 143 columns, which was reduced to 1892 rows and 32 columns.