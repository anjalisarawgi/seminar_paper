Based on the analysis and findings of this seminar paper, it is evident that though sparsity-based estimators (SBEs) like Lasso regression are good for high-dimensional data, they are also very sensitive to normalization choices. Our experiments agree with the original paper's conclusion that SBEs can be unreliable, especially in high-dimensional settings, where normalization decisions can significantly impact stability of the estimates. However, our tests also suggest that these normalization choices may not drastically affect model explainability but may impact the prediction performance.\\
\\
Our results show that it is important to think carefully about the choice of hyperparameters, normalization methods, and preprocessing methods. To be more confident about these results, further experiments across different datasets and diverse contexts are necessary to evaluate the impacts of these choices and methods on the estimates. Additionally, there is still a need for the development of more robust estimators that are less sensitive to normalization choices and hyperparameter settings. \\
\\
In conclusion, the multitude of choices involved in the application of SBEs-ranging from normalization methods to hyperparameter settings-can significantly impact our estimates, and consequently, the overall results. Each dataset and each choice has the potential to impact the stability and reliability of the estimators. This underlines the need for meticulous considerations in practical applications and highlights the ongoing challenges for ensuring the robustness of SBEs across different contexts.