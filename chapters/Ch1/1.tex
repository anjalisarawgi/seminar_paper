Sparsity-based estimators (SBEs), such as Lasso have gained significant attention in statistics and econometrics due to their ability to handle high dimensional data, where the number of potential predictors often exceeds the number of observations. Traditional estimation techniques like Ordinary Lease Squares (OLS) become impractical in such scenarios. SBEs address this by assuming that only a few of the predictors affect the outcome, allowing for more efficient variable selection and model simplification.\\
\\
However, the assumptions underlying these Sparsity-Based Estimators (SBEs), especially the sparsity assumption have been questioned. The sparisty assumption suggests that only a small subset of the many potential variables affect the outcome, which may not always hold. When this assumption is violated, the reliability of SBEs is compromised. Furthermore, the performance of SBEs is sensitive to various methodological choices, such as the construction of the control matrix, the choice of tuning parameters, and the method of normalization, all of which can introduce bias or reduce efficiency. \\
\\
This seminar paper is centered around the findings of the paper ``The Fragility of Sparsity", which critically examines the robustness of SBEs in various empirical contexts. Specifically, "The Fragility of Sparsity" highlights the sensitivity of SBEs to the assumptions and methodological choices made during model specification, raising concerns about their reliability. In this seminar paper, I begin with the interpretation of the original paper, clarifying its main findings along with some examples for better understanding. \\
\\
Building on this, I investigated these concerns in detail, focusing on the empirical validity of the sparsity assumption and the robustness of SBEs to different normalization choices. To extend the discussion, I conduct experiments applying SBEs to a new dataset to test whether the issues highlighted in "The Fragility of Sparsity" are also true in different contexts. Additionally, I explored the impact of varying data sizes on prediction accuracy and loss in machine learning contexts and analyzed how changes in the number of covariates and the regularization parameter for Lasso influence these estimates. This empirical work aims to validate and explore the generalizability of existing findings and to explore how SBEs perform in real-world data scenarios.

